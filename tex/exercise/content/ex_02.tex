\section{Übertragungsfunktionen}

\subsection{$P_1(s)$}
Die Übertragungsfunktion $P_1(s)$ ist gegeben als
\[
	P_1(s) = \frac{\Omega(s)}{U(s)}
\]
Um diese Übertragungsfunktion herzuleiten muss die ursprüngliche
Differetialgleichung betrachtet werden.
\[
	\dot\omega(t) = x \omega(t) + y u(t) + z \Gamma(t)
\]
Da für diese Übertragungsfunktion $\omega(t)$ von Interesse ist, kann die
Funktion dementsprechend in die explizite Form für $\omega(t)$ umgeformt
werden. Dies ist mittels der Integration nach der Zeit $t$ zu bewerkstelligen.
\[
	\omega(t) =
		x \int\omega(t)\,dt
		+ y \int U(t)\,dt 
		+ z \int\Gamma(t)\,dt
\]
Um nun diese Differetialgleichung in den Spektralbereich zu überführen,
können der Linearitätssatz und der Integrationssatz der Laplace-Transformation
genutzt werden.
\[
	\Omega(s) =
		x \frac{1}{s} \Omega(s) 
		+ y \frac{1}{s} U(s)
		+ z \frac{1}{s} \Gamma(s) 
\]
Die nun vorliegende Gleichung im Spektralbereich ist so umzuformen, dass diese
als Qoutient von Ein- und Ausgangsgrösse dargestellt werden kann. In einem
ersten Schritt sind die entsprechenden Termen zu faktorisieren.
\[
	\Omega(s) - x \frac{1}{s} \Omega(s) =
		y \frac{1}{s} U(s) + z \frac{1}{s} \Gamma(s)
\]
\[
	\Omega(s) \left[ 1 - x \frac{1}{s} \right] =
		U(s) \left[ y \frac{1}{s} \right]
		+ \Gamma(s) \left[ z \frac{1}{s}\right]
\]
Betrachtet man die nun vorliegende Gleichung, so stellt man fest, dass eine
Summe vorliegt welche die Bildung des gewünschten Quotienten verhindert.
Betrachtet man diese Gleichung genauer und führt eine Identifikation durch, so
stellt sich heraus, dass einer der Summanden nicht zum relevanten Signalpfad
gehört. Somit kann dieser Teil der Gleichung gestrichen werden
(``\emph{single input, single output}'').
\[
	\underbrace{\Omega(s) \left[ 1 - x \frac{1}{s} \right]}_{Ausgang} =
		\underbrace{U(s) \left[ y \frac{1}{s} \right]}_{Eingang 1}
		+ \underbrace{\Gamma(s) \left[ z \frac{1}{s}\right]}_{Eingang 2}
	\qquad \Gamma(s) = 0 \Rightarrow
		\Omega(s) \left[ 1 - x \frac{1}{s} \right] =
		U(s) \left[ y \frac{1}{s} \right]
\]
Mit dieser Reduktion lässt sich nun die Übertragungsfunktion $P_1(s)$
formulieren.
\[
	P_1(s) = \frac{\Omega(s)}{U(s)} =
		\frac{y \frac{1}{s}}{1 - x \frac{1}{s}} = 
		\frac{\frac{1}{s} y}{\frac{1}{s} (s-x)} =
		\frac{y}{s-x}
\]


\subsection{$P_2(s)$}
