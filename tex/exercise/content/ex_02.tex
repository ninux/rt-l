\section{Übertragungsfunktionen}
\subsection{$P_1(s)$}
Es soll die Übertragungsfunktion hergeleitet werden für $P_1(s)$ mit
\[
	P_1(s) = \frac{\Omega(s)}{U(s)},
	\qquad
	\omega(t) \laplace \Omega(s),
	u(t) \laplace U(s)
\]
Für die Herleitung können die beiden Funktionen $\omega(t)$ und $u(t)$ mit
den Transformationssätzen den Laplace-Transformation überfürt werden vom
Zeitbereich in den Spektralbereich.
\subsubsection{Laplace-Tranformierte von $\omega(t)$}\label{sec:ex_2a}
\[
	\omega(t)
	= x \int\omega(t)\,dt
		+ y \int u(t)\,dt
		+ z \int \Gamma_l(t)\,dt
\]
Die Funktion lässt sich mit dem Linearitätssatz und dem Integrationssatz
überführen in den Spektralbereich zu.
\[
	\Omega(s)
	= x \frac{1}{s} \Omega(s)
		+ y \frac{1}{s} U(s)
		+ z \frac{1}{s} \Gamma_l(s)
	= \frac{1}{s} \left( x \Omega(s) + y U(s) + z \Gamma(s) \right)		
\]
Nun kann die Funktion für $\Omega(s)$ explizit ufgestellt werden.
\[
	\Omega(s) - x \frac{1}{s} \Omega(s)
	= y \frac{1}{s} U(s) + z \frac{1}{s} \Gamma_l(s)
\]
\[
	\Omega(s) \left( 1 - x \frac{1}{s} \right)
	= y \frac{1}{s} U(s) + z \frac{1}{s} \Gamma_l(s) 
\]
\[
	\Omega(s)
	= \frac{
			y \frac{1}{s} U(s) + z \frac{1}{s} \Gamma_l(s)
		}{
			1 - x \frac{1}{s}
		}
	= \frac{
			\frac{1}{s} \left( y U(s) + z \Gamma_l(s) \right)
		}{
			1 - x \frac{1}{s}
		}
	= \frac{1}{s} \left( \frac{
			y U(s) + z \Gamma_l(s)	
		}{
			1 - x \frac{1}{s}
		} \right)
\]

\subsubsection{Laplace-Tranformierte von $u(t)$}
Aus der ursprünglichen Differentialgleichung ergibt sich die Funktion
$u(t)$ zu
\[
	u(t)
	= \frac{1}{y} \left( \dot\omega(t)
		- x \omega(t)
		- z \Gamma(t) \right)
\]
Für die Überführung in den Spektralbereich kann der Linearitätssatz und der
Differentiationssatz der Laplace-Transformation verwendet werden.
\[
	U(s)
	= \frac{1}{y} \left( s \Omega(s)
		-  x \Omega(s)
		-  z \Omega(s) \right) - u(0^-)
\]

\subsubsection{Zusammensetzen der Laplace-Trasformierten}
\[
	P_1(s) =
		\frac{
			\frac{1}{s} \left(
				\frac{y U(s) + z \Gamma_l(s)}{1 - x \frac{1}{s}}
			\right)
		}{
			\frac{1}{y} \left(
				s \Omega(s)
				- x \Omega(s)
				- z \Gamma(s) 
			\right) - u(0^-)
		}
\]

\subsection{$P_2(s)$}
Es soll die Übertragungsfunktion hergeleitet werden für $P_2(s)$ mit
\[
	P_2(s) = \frac{\Omega(s)}{\Gamma_l(s)}
\]

\subsubsection{Laplace-Tranformierte von $\omega(t)$}
Siehe \ref{sec:ex_2a}.

\subsubsection{Laplace-Tranformierte von $\Gamma_l(t)$}
Aus der ursprünglichen Differentialgleichung ergibt sich die Funktion
$\Gamma_l(t)$ zu
\[
	\Gamma_l(t)
	= \frac{1}{z} \left( \dot\omega(t) - x \omega(t) - y U(t) \right)
\]
Für die Überführung in den Spektralbereich kann der Linearitätssatz und der
Differentiationssatz der Laplace-Transformation verwendet werden.
\[
	\Gamma_l(s)
	= \frac{1}{z} \left(
		s \Omega(s)
		- u(0^-)
		- x \Omega(s)
		- y U(s)
	\right)
\]

\subsubsection{Zusammensetzen der Laplace-Trasformierten}
\[
	P_2(s) = \frac{\Omega(s)}{\Gamma_l(s)}
	= \frac{
			\frac{1}{s} \left(
				\frac{y U(s) + z \Gamma_l(s)}{1 - x \frac{1}{s}}
			\right)
		}{
			\frac{1}{z} \left(
				s \Omega(s) - \omega(0^-) - x \Omega(s) - y U(s)
			\right)
		}
\]
