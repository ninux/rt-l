\section{Übertragungsfunktionen programmieren}

Die Übertragungsfunktionen sind für die Prüfung der Korrektheit
parallel zum Wirkungsplan in ein Simulink-Modell eingebaut wie in der
Abbildung \ref{fig:ex_03_model} dargestellt.

\begin{figure}[h!]
	\centering
	\includegraphics[scale=1]{../../matlab/exercise/ex_03/ex_03_model.pdf}
	\caption{Simulink-Modell mit den Übertragungsfunktionen parallel zum 
		Wirkungsplan zur Differentialgleichung.}
	\label{fig:ex_03_model}
\end{figure}

So konnte ein simulultaner Vergleich der Signale durchgeführt werden, um
die Korrektheit der Übertragungsfunktionen sicherzustellen. Das Ergebnis
der Sprungantwort ist in der Abbildung \ref{fig:ex_03_plot} dargestellt.

\begin{figure}[h!]
	\centering
	\includegraphics[scale=1]{../../matlab/exercise/ex_03/ex_03_plot.pdf}
	\caption{Plot der Ausgangsgrösse von Wirkungsplan und
		Übertragungsfunktion}
	\label{fig:ex_03_plot}
\end{figure}

Sämtliche Konfigurationen von Anfangsbedingungen und Belastungen haben das
gleiche Ergebnis geliefert für Wirkungsplan und Übertragungsfunktion.
