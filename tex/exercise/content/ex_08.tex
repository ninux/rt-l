\section{Steuerung}

\subsection{Steuerung als Inversion der Strecke}
Für die vorliegende Strecke soll geprüft werden, ob eine Steuerung als
Inversion der Strecke gewäht werden kann. Die Stecke hat die
Übertragungsfunktion
\[
	P_1(s) = \frac{y}{s-x} = \frac{400}{s + 20.4}
\]
Nimmt man für die Steuerung die Inversion dieser Strecke, so ergibt sich
\[
	S_t(s) = \frac{s-x}{y} = \frac{s + 20.4}{400}
\]
Die Übertragungsfunktion für den Signalpfad ergibt somit
\[
	G_0(s) = S_t(s) \cdot P_1(s) = P_1^{-1}(s) \cdot P_1(s) = 1
\]
Betrachtet man die beiden Übertragungsglieder, so stellt man fest, dass 
$P_1(s)$ eine Tiefpass und $S_t(s)$ einen Hochpass darstellt. Ist solch
eine Steuerung realisierbar? Um dies zu klären muss die Übertragungsfunktion
der Steuerung genauer betrachtet werden.
\[
	S_t(s) = \frac{s-x}{y}
\]
Um die Signale im Zeitbereich zu betrachten, muss die Übertragungsfunktion
mit der inversen Laplace-Transformation aus dem Spektralbereich in den
Zeitbereich überführt werden.
\[
	\mathcal{L}^{-1} \left\{ S_t(s) \right\}
	= \left\{ \frac{s-x}{y} \right\}
\]
Mit dem Linearitätssatz lässt sich die Transformation vereinfachen zu
\[
	\frac{1}{y} \cdot \mathcal{L}^{-1} \left\{ s-x \right\}
	\Rightarrow
	\frac{1}{y} \cdot \left(
		\underbrace{\mathcal{L}^{-1} \left\{ s \right\}}_{
			\begin{subarray}{c}Differen\\-tiator\end{subarray}}
		- \mathcal{L}^{-1} \left\{ x \right\}
	\right)
\]
Die Rücktransformierte der Übertragungsfunktion ergibt demnach
\[
	s_t(t) = \frac{1}{y} \left( \dot\delta(t) - x\delta(t) \right)
\]
Diese Steuerung lässt sich so sicherlich nicht realisieren, da dazu ein
Diracstoss $\delta(t)$ als auch dessen Ableitung $\dot\delta(t)$ notwendig
ist.

\subsection{Proportionalsteuerung}
Statt der Steuerung als Inversion der Strecke wird nun eine Steuerung
implementiert mit einem reinem Proportionalglied, welches exakt dem 
Proportionalanteil der Strecke entspricht.
\[
	S_t(s) = K_p \qquad  P_1(s) = \frac{y}{s-x} = \frac{K_p}{s-x}
\]

\subsection{Simulation und Bewertung der Proportionalsteuerung}
