\section{Steuerung}

\subsection{Steuerung als Inversion der Strecke}
Für die vorliegende Strecke soll geprüft werden, ob eine Steuerung als
Inversion der Strecke gewäht werden kann. Die Stecke hat die
Übertragungsfunktion
\[
	P_1(s) = \frac{y}{s-x} = \frac{400}{s + 20.4}
\]
Nimmt man für die Steuerung die Inversion dieser Strecke, so ergibt sich
\[
	S_t(s) = \frac{s-x}{y} = \frac{s + 20.4}{400}
\]
Die Übertragungsfunktion für den Signalpfad ergibt somit
\[
	G_0(s) = S_t(s) \cdot P_1(s) = P_1^{-1}(s) \cdot P_1(s) = 1
\]
Betrachtet man die beiden Übertragungsglieder, so stellt man fest, dass 
$P_1(s)$ eine Tiefpass und $S_t(s)$ einen Hochpass darstellt. Ist solch
eine Steuerung realisierbar? Um dies zu klären muss die Übertragungsfunktion
der Steuerung genauer betrachtet werden.
\[
	S_t(s) = \frac{s-x}{y}
\]
Um die Signale im Zeitbereich zu betrachten, muss die Übertragungsfunktion
mit der inversen Laplace-Transformation aus dem Spektralbereich in den
Zeitbereich überführt werden.
\[
	\mathcal{L}^{-1} \left\{ S_t(s) \right\}
	= \left\{ \frac{s-x}{y} \right\}
\]
Mit dem Linearitätssatz lässt sich die Transformation vereinfachen zu
\[
	\frac{1}{y} \cdot \mathcal{L}^{-1} \left\{ s-x \right\}
	\Rightarrow
	\frac{1}{y} \cdot \left(
		\underbrace{\mathcal{L}^{-1} \left\{ s \right\}}_{
			\begin{subarray}{c}Differen\\-tiator\end{subarray}}
		- \mathcal{L}^{-1} \left\{ x \right\}
	\right)
\]
Die Rücktransformierte der Übertragungsfunktion ergibt demnach
\[
	s_t(t) = \frac{1}{y} \left( \dot\delta(t) - x\delta(t) \right)
\]
Diese Steuerung lässt sich so sicherlich nicht realisieren, da dazu ein
Diracstoss $\delta(t)$ als auch dessen Ableitung $\dot\delta(t)$ notwendig
ist.

\subsection{Proportionalsteuerung}
Statt der Steuerung als Inversion der Strecke wird nun eine Steuerung
implementiert mit einem reinem Proportionalglied, welches exakt dem 
Proportionalanteil der Strecke entspricht.
\[
	S_t(s) = K_p \qquad  P_1(s) = \frac{y}{s-x} = \frac{K_p}{s-x}
\]

\subsubsection{MATLAB/Simulink-Modell der Steuerung}

\lstinputlisting{../../matlab/exercise/ex_08/ex_08_init.m}

\begin{figure}[h!]
	\centering
	\includegraphics[scale=1]{../../matlab/exercise/ex_08/ex_08_model.pdf}
	\caption{Simulink-Modell der Proportionalsteuerung}
\end{figure}

\subsection{Analyse der Proportionalsteuerung}

\subsubsection{Unterschiedliche Proportionalanteile von $S_t(s)$ und $P_1(s)$}
Die Auslegung des Proportionalitätsgliedes der Steuerung $S_t(s)$ hat direkt
proportionalen Einfluss auf die Ausgangsgrösse. Wird diese Höher oder tiefer
gewählt als jene der Strecke, so ergibt sich ein statischer Regelfehler.

\subsubsection{Angelegtes Lastmoment $\Gamma \neq 0$}
Ist eim Lastmoment angelget $\Gamma \neq 0$ so stellt sich dadurch ein Offest
ein am Ausgang. Dieser Offset ist von der Steuerung nicht berücksichtigt und
führt zu einem massiven statischen Regelfehler.

\subsubsection{Fazit}
Eine Steuerung ist nicht geeignet für dieses System da zum einen ein
anliegendes Lastmoment unberücksichtigt bliebt (statischer Regelfehler bzw.
Offset) und über den Signalpfad nur eine lineare bzw. proportionale
Verstärkung der Eingangsgrösse stattfindet. Diese überlegert sich linear mit
den Einwirkungen des anliegenden Lastmoments.
