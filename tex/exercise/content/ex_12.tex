\section{Übertragungsfunktion}
Die Übertragungsfunktion des relevanten Signalpfades für die Referenzdrezhal
ergit sich zu
\[
	L(s) = C(s) \cdot P_1(s) = K_p \cdot \frac{y}{s-x}
\]
Diese Funktion kann so umgeformt werden, dass das PT$_1$-Glied ($P_1(s)$)
in der Art normiert wird, dass der Nenner die Form $(1 \pm k \cdot s)$ erhält.
\[
	L(s) = K_p \cdot \frac{y}{s-x} =
	K_p \cdot \frac{\frac{y}{x}}{\frac{s-x}{x}} =
	K_p \cdot \frac{\frac{y}{x}}{\frac{1}{x} \cdot s -1}
\]
Führt man die zugehörigen Resubstitutionen durch so ergibt sich die explizite
Übertragungsfunktion.
\[
	K_p := \frac{K}{K^2 + \alpha R}, \qquad
	y := \frac{K}{J R}, \qquad
	x := -\frac{1}{J} \left( \alpha + \frac{K^2}{R} \right)
\]
\[
	L(s) 
	= \frac{K}{K^2 + \alpha R} \cdot \frac{
			-\frac{K}{K^2 + \alpha R}
		}{
			-1 - \frac{1}{\frac{1}{J} \left(
				\alpha + \frac{K^2}{R}
			\right)} \cdot s
		}
	= \frac{K}{K^2 + \alpha R} \cdot \frac{
			\frac{K}{K^2 + \alpha R}
		}{
			1 + \frac{1}{\frac{1}{J} \left(
				\alpha + \frac{K^2}{R}
			\right)} \cdot s
		}
	= 19.61 \cdot \frac{19.61}{1 + 0.049 \cdot s}
\]
