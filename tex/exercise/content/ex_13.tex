\section{Stabilitätsreserve}
Für das Bestimmen der Stabilitätsreserve sollen zwei Wertebereiche für den
Regler gewählt werden.
\[
	K_p  = \left\{
		\begin{array}{l}
			K_p < \frac{1}{K_g} \\
			K_p > \frac{1}{K_g}
		\end{array}
	\right.
\]
Für diese Aufgabe wird angenommen, dass $K_g$ den Proportionalanteil
der normierten Übertragungsfunktion der Stecke darstellt.
\[
	K_g = \frac{y}{x} = \frac{K}{K^2 + \alpha R}
\]
Da keine expliziten Werte gegeben sind, wird zunächt der Grenzfall
betrachtet mit
\[
	K_p = \frac{1}{K_g} = \frac{K^2 + \alpha R}{K}
\]
Die Übertragungsfunktion für den relevanten Signalpfad ergibt sich somit zu
\[
	L(s) = \frac{1}{K_g} \cdot \frac{K_g}{1 + T_1 \cdot s}
	= \frac{1}{1 + T_1 \cdot s} 
	= \frac{1}{1 + \frac{1}{\frac{1}{J} \cdot \left(
		\alpha + \frac{K^2}{R} \right) } \cdot s }
\]
Das daraus resultierende Glied stellt einen Tiefpass dar (PT$_1$-Glied).
Dieses hat einen resultierenden Proportionalitätsfaktor von 0, da der 
Regler $K_g$ reziprok als Faktor inplementiert. Somit ergibt sich der
Verlauf eines Tiefpasses mit $G_0 = 0$. Eine Skalierung beim Regler
verändert das Verhalten der Übertragung in der Weise, dass legiglich die
DC-Verstärkung abgehiben oder abgesenkt wird. Dies ist in der Abbildung
\ref{fig:ex_13_bode} dargestellt.
\begin{figure}[h!]
	\centering
	\includegraphics[width=\textwidth]{../../matlab/exercise/ex_13/ex_13_bode.pdf}
	\caption{Bode-Diagramm des Grenzfalls}
	\label{fig:ex_13_bode}
\end{figure}
Die Stabilitätsreserven (Amplitudenreserve $A_R$ und Phasenreserve
$\varphi_R$) sind dirket deutbar aus dem Plot \ref{fig:ex_13_bode}.

\subsection{Grenzfall $K_p = {K_g}^{-1}$}
Für den Grenzfall wird deutlich, dass der Durchtritt ($G = 0$dB) schon
bei einer Phasenverschiebung von $\varphi > 0^{\circ}$ erfolgt.
Da es sich um einen einfachen Tiefpass (Tiefpass erster Ordnung bzw.
PT$_1$-Glied) handelt, wird dessen Phase
asymptotisch gegen $-90^{\circ}$ gehen. Somit ergibt sich eine unendliche
Amplitudenreserve $A_R = \infty$. Die Phasenreserve ist dabei exakt
$\varphi_R = 180^{\circ}$, da der Durchtritt bereits bei
$\varphi > 0^{\circ}$ erfolgt.

\subsection{$K_p > {K_g}^{-1}$}
Wird ein Proportionalitätsfaktor $K_p > {K_g}^{-1}$ gewählt, so wird der
Amplitudengang des Übertragungsgliedes linear nach oben geschoben im
Bode-Diagramm. Für die Stabilität bedeutet dies, dass der Durchtritt
später erfolgt und die Phasendrehung bereits weiter propagiert ist.
Für die Amplitudenreserve spielt dies keine Rolle, da hier ein System
erster Ordnung vorliegt (PT$_1$-Glied) mit einer maximalen Phasendrehung
von $\varphi = -90^{\circ}$. Die Phasenreserve jedoch wird durch den
Versatz der Durchtriffsfrequenz tangiert. Diese kann somit bis auf
$90^{\circ}$ reduziert werden.

\subsection{$K_p < {K_g}^{-1}$}
Wird ein Proportionalitätsfaktor $K_p < {K_g}^{-1}$ gewählt, so wird der
Amplitudengang des Übertragungsgliedes linear nach unten geschoben im
Bode-Diagramm mit $G_0 < 0$dB. Für die Stabilität bedeutet dies, dass der
Durchtritt nie erfolgt. Für die Amplitudenreserve spielt dies keine Rolle,
da hier ein System erster Ordnung vorliegt (PT$_1$-Glied) mit einer
maximalen Phasendrehung von $\varphi = -90^{\circ}$. Die Phasenreserve
wird so aber unendlich gross, da kein Durchtritt vorliegt ($G_0 = 0$dB).

\subsection{MATLAB Script und Simulationsergebnisse}

\lstinputlisting{../../matlab/exercise/ex_13/ex_13_init.m}
\lstinputlisting{../../matlab/exercise/ex_13/ex_13_bode.m}
\lstinputlisting{../../matlab/exercise/ex_13/results.txt}

