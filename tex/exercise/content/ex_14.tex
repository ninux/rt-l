\section{Geschlossener Regelkreis}

\subsection{Führungsübertragungsfunktion $G_F(s)$}
Die Führungsübertragungsfunktion setzt sich Zusammen aus dem
direkten Signalpfad von Eingang zu Ausgang des Systems und dem
Kreispfad.
\[
	G_F(s) = \frac{C(s) \cdot P_1(s)}{1 + C(s) \cdot P_1(s)}
\]
Resubstituiert man die einzelnen Glieder mit ihren normierten Komponenten
so ergibt sich die explizite Führungsübertragungsfunktion.
\[
	G_F(s)
	= \frac{C(s) \cdot P_1(s)}{1 + C(s) \cdot P_1(s)}
	= \frac{
		K_p \cdot \frac{K_{g_1}}{1+T_1\cdot s}
	}{
		1 + K_p \cdot \frac{K_{g_1}}{1+T_1\cdot s}	
	}
	= \frac{K_p \cdot K_{g_1}}{1 + T_1 \cdot s + K_p \cdot K_{g_1}}
\]

\subsection{Störübertragungsfunktion $G_S(s)$}
Die Störübertragungsfunktion setzt sich zusammen aus dem direkten Pfad
von der Störquelle hin zum Ausgang (denn ein Kreispfad ist im vorliegenden
Modell nicht vorhanden).
\[
	G_S(s) = P_2(s)
\]
Resubstituiert man die Übertragungsfunktion $P_2(s)$ mit den normierten
Komponenten so ergibt sich die explizite Störübertragungsfunktion.
\[
	G_S(s) = P_2(s) = \frac{z}{s-x} = \frac{K_{g_2}}{1 + T_2 \cdot s}
\]
