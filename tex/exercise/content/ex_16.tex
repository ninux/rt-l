\section{Stationäre Regeldifferenz}
Die stationäre Regeldifferenz lässt sich mittels des Endwertsatzes von
Laplace ermitteln.
\[
	\omega(\infty)
	= \lim_{s \rightarrow 0} s \cdot G_F(s) \cdot \Omega(s)	
\]
Hierbei ist $G_F(s)$ die Führungsübertragungsfunktion und $\Omega(s)$ die
Laplace-Transformierte des Eingangssignals. Dieses soll hier als
Einheitssprung mit der Amplitude 1 gegeben sein.
\[
	\omega(t)
	= 1 \cdot \sigma(t) \quad \laplace \quad \Omega(s) = \frac{1}{s}
\]
Wendet man den Endwertsatz an auf die gegebenen Funktionen so ergibt sich
\[
	\omega(\infty)
	= \lim_{s \rightarrow 0} s \cdot G_F(s) \cdot \Omega(s)	
	= s \cdot \frac{
		K_p \cdot K_{g_1}
	}{
		1 + T_1 \cdot s + K_p + K_{g_1}
	} \cdot \frac{1}{s}
	= \frac{K_p \cdot K_{g_1}}{1 + K_p \cdot K_{g_1}}
\]
Nun wird deutlich, dass die Regeldifferenz direkt von der Auslegung der
Propotionalitätskonstante $K_p$ des Reglers abhängt. Wählt man beispielsweise
ein $K_p$, so dass dieses dem reziproken Wert des $K_{g_1}$ entspricht, so
ergibt sich ein Propotionalitätfaktor von $K_p \cdot K_{g_1} = 1$. Setzt man
dies ein, so ergibt sich ein Endwert von 
\[
	\omega(\infty) = \frac{1}{1+1} = \frac{1}{2} = 0.5 = 50\%	
\]
Wird das $K_p$ hingegen kleiner ${K_{g_1}}^{-1}$ gewählt, so wird der Endwert
nochmals kleiner als bei der Gleichheit.
\[
	K_p < \frac{1}{K_{g_1}}
	\Rightarrow K_p \cdot K_{g_1} < 1 
		\Rightarrow \omega(\infty) \xrightarrow{K_p \rightarrow 0} 0
\]
Wird das $K_p$ hingegen grösser gewählt, so wird der Endwert höher.
\[
	K_p > \frac{1}{K_{g_1}}
	\Rightarrow K_p \cdot K_{g_1} > 1
		\Rightarrow \omega(\infty) \xrightarrow{K_p \rightarrow \infty} 1
\]

\subsection{Prüdung der Endwerte mit MATLAB/Simulink}
Die Simulation des Systems mit Simulink bestätigt die Annahmen, welche sich
durch den Endwertsatz ergeben.
