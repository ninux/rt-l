\section{Stationäre Regeldifferenz}
Die stationäre Regeldifferenz lässt sich mittels des Endwertsatzes von
Laplace ermitteln.
\[
	\omega(\infty)
	= \lim_{s \rightarrow 0} s \cdot G_F(s) \cdot \Omega(s)	
\]
Hierbei ist $G_F(s)$ die Führungsübertragungsfunktion und $\Omega(s)$ die
Laplace-Transformierte des Eingangssignals. Dieses soll hier als
Einheitssprung mit der Amplitude 1 gegeben sein.
\[
	\omega(t)
	= 1 \cdot \sigma(t) \quad \laplace \quad \Omega(s) = \frac{1}{s}
\]
Wendet man den Endwertsatz an auf die gegebenen Funktionen so ergibt sich
\[
	\omega(\infty)
	= \lim_{s \rightarrow 0} s \cdot G_F(s) \cdot \Omega(s)	
	= s \cdot \frac{
		K_p \cdot K_{g_1}
	}{
		1 + T_1 \cdot s + K_p + K_{g_1}
	} \cdot \frac{1}{s}
	= \frac{K_p \cdot K_{g_1}}{1 + K_p \cdot K_{g_1}}
\]
