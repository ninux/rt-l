\section{Reglerauslegung}
Die Auslegung des Proportionalitätsfaktors $K_p$ des Reglers, unter
Berücksichtigung der Stabilitätsanalyse und des stationären Regelfehlers,
sollte nach Möglichkeit hoch ausfallen. Dies beweikt, dass der stationäre
Regelfehler gegen 0 strebt. Die Stabilität ist hiervon nicht tangiert
(vorausgesetzt das Lastmoment ist $\Gamma = 0$).

Wird ein Lastmoment $\Gamma \neq 0$ angelegt, so gilt die obige aussage
nicht. Denn der Einfluss der Störgrösse wird überlagert auf die
Ausgangsgrösse. Je höher der Proportionalitätsfaktor gewählt wird, desto
höher ist die Schwingneigung. 

Simulationen mit dem Modell in MATLAB/Simulink zeigen, dass sich mit einem
hohen $K_p$ und einem $\Gamma \neq 0$ nie ein stationärer Wert am Ausgang
einstellt bei konstanter Führungsgrösse. Wählt man das $K_p$ jedoch kleiner,
so wird die Amplitude und Frequenz der Schwingungen heruntergesetzt.

Betrachtet man die Störübertragungsfunktion im Bode-Diagramm (siehe Abbildung
\ref{fig:ex_17_bode_p2}), so erkennt man, dass dort eine massive
DC-Verstärkung $G_0 > 45$dB vorliegt. Zuden ist zu erkennen, dass die
Phase nicht bei $0^{\circ}$ beginnt, sondern bei $180^{\circ}$. Dies ist so,
da die Störung negativ wirkt, also der Drehzahl $\omega$ entgegengesetzt.

\begin{figure}[h!]
	\centering
	\includegraphics[width=1\textwidth]{../../matlab/exercise/ex_17/ex_17_bode_p2.pdf}
	\caption{Bode-Diagramm der Störübertragungsfunktion $P_2(s)$}
	\label{fig:ex_17_bode_p2}
\end{figure}
