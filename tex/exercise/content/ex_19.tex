\section{Parameter des PI-Reglers}
Für die Betrachtung des offenen Regelkreises mit dem PI-Regler wird dieser
in einem ersten Schritt für den Grenzfall $T_i = \tau_g$ ausgelegt.
\[
	\tau_g := \frac{J R}{K^2 + \alpha R} \approx 0.05 
\]
Somit ergibt sich die folgende Übertragungsfunktion für den offenen
Regelkreis.
\[
	L(s)
	= C(s) \cdot P_1(s)
	= K_p \cdot \left(
		1 + \frac{1}{T_i s}
	\right) \cdot \left(
		\frac{y}{s-x}
	\right)
	= \underbrace{\left(
		K_p + \frac{
			K_p \left(K^2 + \alpha R\right)
		}{JR} \cdot \frac{1}{s}
	\right)}_{\text{PI-Regler}} \cdot \underbrace{\left(
		\frac{
			\frac{K}{R \left(\alpha + \frac{K^2}{R}\right)}
		}{
			1 + \frac{J}{\alpha + \frac{K^2}{R}} \cdot s
		}
	\right)}_{\text{PT$_1$-Strecke}}
\]
Aus der oben aufgeführten Übertragungsfunktion $L(s)$ lassen sich nun
direkt die Kennwerte für das Bode-Diagramm auslesen. Der Durchtrittspunkt
$\omega_D$ ist gegeben durch den Faktor zum Integralanteil $s^{-1}$.
\[
	\text{I}: \qquad \underbrace{k_i}_{\omega_D} \cdot \frac{1}{s}
\]
Dieser ist aus obiger Übertragungsfunktion direkt auszulesen.
\[
	L(s) = \left(
		K_p + \underbrace{\frac{
			K_p \left(K^2 + \alpha R\right)}{JR}}_{\omega_D}
		\cdot \frac{1}{s}
	\right) \cdot \left(
		\frac{
			\frac{K}{R \left(\alpha + \frac{K^2}{R}\right)}
		}{
			1 + \frac{J}{\alpha + \frac{K^2}{R}} \cdot s
		}
	\right)
\]
Der Summand $K_p$ bildet dabei lediglich die Limitierung für die Dämpfung.
D.h., dass die Verstärkung asymtotisch zu 0 geht statt logarithmisch (linear
im Bode-Diagramm mit logarithmischer Darstellung) nach $\infty$, wie dies
beim idealen Integrator der Fall ist (hier liegt ein PI-Glied vor).
Dieser P-Anteil bewirkt auch eine Phasenhebung von $+90^{\circ}$ (ausgehend
von $-90^{\circ}$ für $\omega \rightarrow 0$ hervorgerufen durch den
Integrator). Die Bode-Diagramme für diese Auslegung sind in der Abbildung
\ref{fig:ex_19_bode_border} dargestellt. In dieser Abbildung ist deutlich
zu erkennen, dass sich die Phasenhebung des PI-Gliedes und die Phasensenkung
des PT$_1$-Gliedes gegenseitig aufheben, so dass eine Phase von
$\varphi = 90^{\circ}$ reusltiert. Auch der Amplitudengang wird durch die
Kombination von PI- und PT$_1$-Glied so verändert, dass ein Amplitudengang
eines idealen Integrators resultiert. Die Abbildung
\ref{fig:ex_19_bode_variants} zeigt den Einfluss der Auslegung von $T_i$
im Verhältnis zu $\tau_g$.

\begin{figure}[h!]
	\centering
	\includegraphics[width=1\textwidth]{../../matlab/exercise/ex_19/ex_19_bode.pdf}
	\caption{Bode-Diagramm des Reglers $C(s)$, der Strecke $P_1(s)$ und des
		direkten Signalpfades $L(s)$.}
	\label{fig:ex_19_bode_border}
\end{figure}

\begin{figure}[h!]
	\centering
	\includegraphics[width=1\textwidth]{../../matlab/exercise/ex_19/ex_19_bode_02.pdf}
	\caption{Bode-Diagramm von $L(s)$ für $T_i = \tau_g$ und $T_i \neq \tau_g$.}
	\label{fig:ex_19_bode_variants}
\end{figure}
