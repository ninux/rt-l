\section{Stationäre Regeldifferenz des PI-Reglers}
Um die stationäre Regeldifferenz zu ermitteln, muss zunächst die
Übertragungsfunktion formuliert werden.
\[
	G_F(s) = \frac{G_0(s)}{1+G_0(s)}
\]
\[
	G_0(s)
	= C(s) \cdot P_1(s)
	= K_p \left( 1 + \frac{1}{T_i s} \right) \frac{y}{s-x}
	= \frac{K_p(T_i s + 1) y}{T_i s (s-x)}
\]
\[
	G_F(s)
	= \frac{G_0(s)}{1 + G_0(s)}
	= \frac{
		\left( \frac{K_p(T_i s + 1) y}{T_i s (s-x)} \right)
	}{
		1 + \left( \frac{K_p(T_i s + 1) y}{T_i s (s-x)} \right)
	}
	= \frac{
		K_p y (T_i s + 1)
	}{
		T_i(s^2 - sx) + K_p y (T_i s + 1)
	}
\]
\[
	\omega(\infty)
	= \lim_{s \rightarrow 0} \left(
		\underbrace{s \cdot \Omega(s)}_{1} \cdot G_F(s)
	\right)
	= \lim_{s \rightarrow 0} G_F(s)
	= K_p \cdot y \cdot \lim_{s \rightarrow 0} \left(
		\frac{
			\overbrace{T_i s}^{0} + 1
		}{
			\underbrace{T_i(s^2 - sx)}_{0} 
			+ \underbrace{K_p y (T_i s + 1)}_{K_p y}
		}
	\right)
	= 1
\]
