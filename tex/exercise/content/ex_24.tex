\section{Signalbegrenzung}

\subsection{MATLAB/Simulink-Modell}

\begin{figure}[h!]
	\centering
	\includegraphics[width=1\textwidth]{../../matlab/exercise/ex_24/ex_24_model.pdf}
	\caption{Simulink-Modell des geregelten Motors mit Signalbegrenzung}
\end{figure}

\subsection{Simulationsergebnisse}
Je niedriger $K_p$ gewählt wird, desto träger ist der Ausgleichsvorgang.
Die Drehzahl reagiert mit einer $e$-Kurve auf einen Sprung. Rippel oder
Schwingungen auf der Drehzahl sind jedoch keine vorhanden. Auch bei der
Eingangsspannung des Motors ist keinerlei Rippel zu beobachten.

Wählt man aber ein hohes $K_p$, so ist der Ausgleichsvorgang linear. Die
Spannung am Motoreneingang zeigt dann jedoch dutliche Rippel bzw.
Schwingungen auf.

\begin{figure}[h!]
	\centering
	\begin{subfigure}{0.45\textwidth}
		\includegraphics[width=1\textwidth]{../../matlab/exercise/ex_24/ex_24_01.pdf}
		\caption{$u(t)$ und $\omega(t)$ bei kleinem $K_p$}
	\end{subfigure}
	\hfill{}
	\begin{subfigure}{0.45\textwidth}
		\includegraphics[width=1\textwidth]{../../matlab/exercise/ex_24/ex_24_02.pdf}
		\caption{$u(t)$ und $\omega(t)$ bei grossen $K_p$}
	\end{subfigure}
	\caption{Simulationen verschiedener $K_p$}
\end{figure}
