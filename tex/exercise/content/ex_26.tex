\section{Signalbegrenzung - Anti-Windup}
Ein Anti-Windup verhindert das ``Überlaufen'' des Integrators, welches eine
Verzögerung verursacht wenn die Führungsgrösse reduziert wird. Die Verzögerung
kommt zustande, wenn eine Limitierung vorherrscht zwischen Integrator und
Stellglied bzw. Strecke. Hat der Integrator einen grösseren Amplitudenbereich
so kann dieser die Eingabe weiter aufintegrieren, da der Fehler durch die
Limitierung hin zu Strecke konstant bleibt -- der Integrator ``läuft über''
(engl. \emph{windup}).

Um diesem Effekt entgegenzuwirken bedarf es der Reduktion des Eingangssignals
am Integrator. Die Reduktion muss dabei in dem Masse stattfinden, so dass
diese nur im limitierenden Fall der Strecke vorkommt. Eine Möglichkeit ist die
Differenz von Integratorausgang und der limitierten Stellgrösse vom Eingang
des Integrators abzuziehen wie in der Abbildung \ref{fig:ex_26_model}
dargestellt.

\subsection{MATLAB/Simulink-Modell}

\begin{figure}[h!]
	\centering
	\includegraphics[width=1\textwidth]{../../matlab/exercise/ex_26/ex_26_model.pdf}
	\caption{Simulink-Modell des geregelten Motors mit Signalbegrenzung
		und Anti-Windup}
	\label{fig:ex_26_model}
\end{figure}

\subsection{Simulationsergebnisse}
Die Simulation zeigt, dass sich die Ergebnisse für die Sprungantwort nicht
ändern mit der Einführung des Anti-Windup (siehe Abbildung \ref{fig:ex_26_01}
und \ref{fig:ex_25_01}). Dies ist auch zu erwarten, da das Überlaufen des
Integrators erst bei der Reduktion der Führungsgrösse zur Geltung kommt. 
Ein deutlicher Unterschied zeigt sich jedoch beim Vergleich der
Impulsantworten (siehe Abbildung \ref{fig:ex_26_11} und \ref{fig:ex_25_01}).
Die Dynamik eines Systems lässt sich also drastisch erhöhen wenn die
Stellgrössenlimitierung im Regler mitberücksichtigt wird.

\begin{figure}[h!]
	\centering
	\begin{subfigure}{0.45\textwidth}
		\includegraphics[width=1\textwidth]{../../matlab/exercise/ex_26/ex_26_01.pdf}
		\caption{Sprungantwort für kleines $K_p$}
		\label{fig:ex_26_01}
	\end{subfigure}
	\hfill{}
	\begin{subfigure}{0.45\textwidth}
		\includegraphics[width=1\textwidth]{../../matlab/exercise/ex_26/ex_26_02.pdf}
		\caption{Sprungantwort für grosses $K_p$}
	\end{subfigure}

	\begin{subfigure}{0.45\textwidth}
		\includegraphics[width=1\textwidth]{../../matlab/exercise/ex_26/ex_26_11.pdf}
		\caption{Impulsantwort für kleines $K_p$}
		\label{fig:ex_26_11}
	\end{subfigure}
	\hfill{}
	\begin{subfigure}{0.45\textwidth}
		\includegraphics[width=1\textwidth]{../../matlab/exercise/ex_26/ex_26_12.pdf}
		\caption{Impulsantwort für grosses $K_p$}
	\end{subfigure}

	\caption{Simulationen verschiedener $K_p$ für das Anti-Windup}

\end{figure}
