\section{Simulation der Vorsteuerung}

\subsection{Simulink-Modell}
\begin{figure}[h!]
	\centering
	\includegraphics[width=1\textwidth]{../../matlab/exercise/ex_28/ex_28_model.pdf}
	\caption{Simulink-Modell mit Vorsteuerung $F(s)$}
\end{figure}

\subsection{Simulationsergebnisse}
Die Simulationen zeigen, dass die Vorsteuerung einen positiven Einfluss hat
auf den Regler. Zum einen verringern sich die Ausschläge der Fehler und zum
anderen wird das System dynamischer (siehe Abbildung \ref{fig:ex_28_11} und
\ref{fig:ex_26_11}).
\begin{figure}[h!]
	\centering
	\begin{subfigure}{0.45\textwidth}
		\includegraphics[width=1\textwidth]{../../matlab/exercise/ex_28/ex_28_01.pdf}
		\caption{Sprungantwort für kleines $K_p$}
		\label{fig:ex_28_01}
	\end{subfigure}
	\hfill{}
	\begin{subfigure}{0.45\textwidth}
		\includegraphics[width=1\textwidth]{../../matlab/exercise/ex_28/ex_28_02.pdf}
		\caption{Sprungantwort für grosses $K_p$}
	\end{subfigure}
	\caption{Simulation der Sprungantworten der Vorsteuerung}
\end{figure}

\begin{figure}[h!]
	\begin{subfigure}{0.45\textwidth}
		\includegraphics[width=1\textwidth]{../../matlab/exercise/ex_28/ex_28_11.pdf}
		\caption{Impulsantwort für kleines $K_p$}
		\label{fig:ex_28_11}
	\end{subfigure}
	\hfill{}
	\begin{subfigure}{0.45\textwidth}
		\includegraphics[width=1\textwidth]{../../matlab/exercise/ex_28/ex_28_12.pdf}
		\caption{Impulsantwort für grosses $K_p$}
	\end{subfigure}
	\caption{Simulation der Impulsantworten der Vorsteuerung}
\end{figure}
